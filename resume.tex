%% start of file `template.tex'.
%% Copyright 2006-2013 Xavier Danaux (xdanaux@gmail.com).
%
% This work may be distributed and/or modified under the
% conditions of the LaTeX Project Public License version 1.3c,
% available at http://www.latex-project.org/lppl/.

\documentclass[11pt,a4paper,sans]{moderncv}       % possible options include font size ('10pt', '11pt' and '12pt'), paper size ('a4paper', 'letterpaper', 'a5paper', 'legalpaper', 'executivepaper' and 'landscape') and font family ('sans' and 'roman')

% moderncv themes
\moderncvstyle{classic}                           % style options are 'casual' (default), 'classic', 'oldstyle' and 'banking'
\moderncvcolor{red}                               % color options 'blue' (default), 'orange', 'green', 'red', 'purple', 'grey' and 'black'
%\renewcommand{\familydefault}{\sfdefault}        % to set the default font; use '\sfdefault' for the default sans serif font, '\rmdefault' for the default roman one, or any tex font name
%\nopagenumbers{}                                 % uncomment to suppress automatic page numbering for CVs longer than one page

% character encoding
\usepackage[utf8]{inputenc}                       % if you are not using xelatex ou lualatex, replace by the encoding you are using
%\usepackage{CJKutf8}                             % if you need to use CJK to typeset your resume in Chinese, Japanese or Korean

% adjust the page margins
\usepackage[scale=0.75]{geometry}
%\setlength{\hintscolumnwidth}{3cm}               % if you want to change the width of the column with the dates
%\setlength{\makecvtitlenamewidth}{10cm}          % for the 'classic' style, if you want to force the width allocated to your name and avoid line breaks. be careful though, the length is normally calculated to avoid any overlap with your personal info; use this at your own typographical risks...

% personal data
\name{Kirill}{Bobyrev}
\title{Computer Scientist}
\address{Moscow}{Russia}{}
\email{kirillbobyrev@gmail.com}
\homepage{omtcvxyz.github.io}
\github{omtcvxyz}
\linkedin{omtcvxyz}

%-------------------------------------------------------------------------------
%            content
%-------------------------------------------------------------------------------
\begin{document}
%-----       resume       ------------------------------------------------------
\makecvtitle

\section{Education}
  \cventry{Class of 2019}{B.Sc.}{}{Moscow Institute of Physics and Technology}{}{}

\section{Experience}
  \cventry{2018}{Software Engineering Intern}{Google}{Munich}{}{}
  \cventry{2016}{Software Engineering
    Intern}{Google}{Munich}{}{Peers: Alexander Kornienko and Manuel
    Klimek, Team: C++ Language Tools
    \begin{itemize}%
      \item Significantly improving clang-rename and gaining users
      \item Designing and prototyping
        \href{https://clang.llvm.org/docs/RefactoringEngine.html}{\textcolor{blue}{clang-refactor}}
        (see
        \href{https://docs.google.com/document/d/1w9IkR0_Gqmd5w4CZ2t_ZDZrNLYVirQPyMS41533HQZE/edit?usp=sharing}{\textcolor{blue}{design
        doc}})
      \item Adding new functionality to clang-tidy, a C++ linter tool
    \end{itemize}}
  \cventry{2015}{Google Summer of Code student}{LLVM
    Community}{Remote}{}{Mentor: Vassil Vassilev\newline{}
    Implementing Code Clone Detection tool in Clang Static Analyzer}
  \cventry{2012-2013}{Junior
    Game Developer}{Subterranean Games}{Remote}{}{Participating in
    \href{http://store.steampowered.com/app/230190}{\textcolor{blue}{War For the
    Overworld}} game development}

\section{Projects}
  \cventry{2015-Present}{LLVM}{}{}{}{Participating in LLVM-related projects and
    working on Clang-based C++ tooling\newline{}}
  \cventry{2016}{VisARTM}{}{}{}{Implementing and maintaining
    \href{https://github.com/bigartm/bigartm}{\textcolor{blue}{BigARTM}}
    Topic Model navigator under supervision of K.V. Vorontsov}

\section{Skills}
  \cvitem{Programming}{C++, Python, Rust, Haskell}
  \cvitem{Software}{Linux, Vim, \LaTeX, Git, Tmux}
  \cvitem{Libraries}{LLVM, NumPy, SciPy, Tensorflow}
  \cvitem{Spoken Languages}{Russian (native), English (fluent), German
          (advanced), Chinese (beginner)}

\nocite{*}
\bibliographystyle{plain}
\bibliography{publications}

\end{document}
