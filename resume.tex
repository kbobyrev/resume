%% Copyright 2006-2013 Xavier Danaux (xdanaux@gmail.com).
%
% This work may be distributed and/or modified under the
% conditions of the LaTeX Project Public License version 1.3c,
% available at http://www.latex-project.org/lppl/.

% possible options include font size ('10pt', '11pt' and '12pt'), paper size
% ('a4paper', 'letterpaper', 'a5paper', 'legalpaper', 'executivepaper' and
% 'landscape') and font family ('sans' and 'roman')
\documentclass[10pt,a4paper,sans]{moderncv}

% moderncv themes
\moderncvstyle{classic}
\moderncvcolor{red}
\nopagenumbers{}

% character encoding
\usepackage[utf8]{inputenc}

% adjust the page margins
\usepackage[scale=0.8]{geometry}

% personal data
\name{Kirill}{Bobyrev}
\title{Computer Scientist}
\address{Moscow, Russia}{}{}
\email{kirillbobyrev@gmail.com}
\homepage{kbobyrev.github.io}
\github{kbobyrev}
\twitter{kirillbobyrev}
\linkedin{kbobyrev}

%-------------------------------------------------------------------------------
% Resume
%-------------------------------------------------------------------------------
\begin{document}

\makecvtitle

\section{Education}
  \cventry{Class of 2019}{B.Sc.}{Moscow, Russia}{Moscow Institute of Physics and
    Technology}{Applied Mathematics}{}

\section{Experience}
  \cventry{2018}{Software Engineering Intern}
    {Google}{Munich}{}{}
  \cventry{2016}{Software Engineering
    Intern}{Google}{Munich}{C++ Language Tools Team}{\begin{itemize}
      \item Fixed a number of existing issues, introduced new functionality
        and gained users for clang-rename
      \item Designed and prototyped clang-refactor (which later split into
        \href{https://clang.llvm.org/docs/RefactoringEngine.html}
        {\textcolor{blue}{RefactoringEngine}},
        \href{https://clang.llvm.org/extra/clangd.html}
        {\textcolor{blue}{clangd}} and few other projects, see
        \href{https://docs.google.com/document/d/1w9IkR0_Gqmd5w4CZ2t_ZDZrNLYVirQPyMS41533HQZE/edit?usp=sharing}
        {\textcolor{blue}{design doc}} for reference)
      \item Added new functionality to clang-tidy, a C++ linter tool
    \end{itemize}}
  \cventry{2015}{Google Summer of Code student}{LLVM
    Community}{Remote}{}{Implemented Code Clone Detection tool in Clang Static
    Analyzer and used it to detect similar code pieces in such projects as Git,
    Vim and OpenSSL:\@ an overview of key results can be found
    \href{https://github.com/kbobyrev/code-clone-detection-llvm-devmtg15-poster\#code-clone-detection-llvm-devmtg15-poster}{\textcolor{blue}{here}}}

  \cventry{2012--2013}{Junior Game Developer}{Subterranean
    Games}{Remote}{}{Supported
    \href{http://store.steampowered.com/app/230190}{\textcolor{blue}{War For
    the Overworld}} game development by implementing infrastructure tools for
    level and game scenarios design during a successful
    \href{https://www.kickstarter.com/projects/subterraneangames/war-for-the-overworld}{\textcolor{blue}{Kickstarter
    Campaign}}}

\section{Projects}
  \cventry{2018--Present}{Quarum}{Prototyping in progress}
    {\url{https://github.com/kbobyrev/quarum}}{}
    {Implementing efficient QR Code Encoder/Decoder in Rust}
  \cventry{2018--Present}{Heathen}{Prototyping in progress}
    {\url{https://github.com/kbobyrev/heathen}}{}
    {Designing and implementing small programming language in C++ using LLVM}
  \cventry{2018}{$\texttt{TD}(\gamma)$}{}
    {\url{https://github.com/kbobyrev/optimization-class-project}}{}
    {Implementing Reinforcement Learning algorithm after the original paper}
  \cventry{2015--Present}{LLVM}{Open Source Compiler Infrastructure Project}{}
    {}{Participating in LLVM-related projects and working on Clang-based C++
       tooling}

\section{Skills}
  \cvitem{Programming}{C++ (advanced), Python (advanced), Rust (intermediate),
                       Haskell (beginner)}
  \cvitem{Software}{Vim, Git, \LaTeX, CMake, LLDB, tmux, Linux}
  \cvitem{Libraries}{LLVM, NumPy, MPI, OpenMP, SciPy, TensorFlow, Boost}
  \cvitem{Interests}{Stochastic Optimization, Reinforcement Learning,
                     Compilers, Static Analysis}
  \cvitem{Languages}{English (fluent), Russian (fluent), German
          (advanced), Chinese (beginner)}

\nocite{*}
\bibliographystyle{plain}
\bibliography{publications}

\end{document}
