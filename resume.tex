%% Copyright 2006-2013 Xavier Danaux (xdanaux@gmail.com).
%
% This work may be distributed and/or modified under the
% conditions of the LaTeX Project Public License version 1.3c,
% available at http://www.latex-project.org/lppl/.

% possible options include font size ('10pt', '11pt' and '12pt'), paper size
% ('a4paper', 'letterpaper', 'a5paper', 'legalpaper', 'executivepaper' and
% 'landscape') and font family ('sans' and 'roman')
\documentclass[10pt,a4paper,sans]{moderncv}

% moderncv themes
\moderncvstyle{classic}
\moderncvcolor{red}
\nopagenumbers{}

% character encoding
\usepackage[utf8]{inputenc}

% adjust the page margins
\usepackage[scale=0.8]{geometry}

% personal data
\name{Kirill}{Bobyrev}
\title{Computer Scientist}
\address{Moscow, Russia}{}{}
\email{kirillbobyrev@gmail.com}
% \homepage{kirillbobyrev.com}
\github{kirillbobyrev}
\twitter{kirillbobyrev}
\linkedin{kirillbobyrev}
\goodreads{kirillbobyrev}

%-------------------------------------------------------------------------------
% Resume
%-------------------------------------------------------------------------------
\begin{document}

\makecvtitle

\section{Education}
  \cventry{Class of 2019}{BSc}{Russia}{Moscow Institute of Physics and
    Technology}{Applied Mathematics and Physics}{}

\section{Experience}
  % TODO(kirillbobyrev): Tell more about tasks performed, partners I have worked
  % with (AMG, etc) and negotiations.
  \cventry{2019}{Data Scientist}{Dbrain}{}{}{
    Implemented Deep Learning models for various Computer Vision tasks
    including Image Classification and Object Segmentation using PyTorch and
    TensorFlow, built Machine Learning pipelines starting from raw data and
    producing necessary predictions on par with state-of-the-art results.}
  \cventry{2018}{Software Engineering Intern}
    {Google}{}{C++ Language Tools Team}
    {Worked on Clangd, Clang-based C++ LSP implementation used by CLion, XCode,
     Eclipse and other IDEs
     \begin{itemize}
       \item Designed and implemented
         \href{https://docs.google.com/document/d/1C-A6PGT6TynyaX4PXyExNMiGmJ2jL1UwV91Kyx11gOI/edit?usp=sharing}
         {\textcolor{blue}{Dex}} --- efficient Clangd symbol index for fuzzy
         matching code completion.
       \item New index implementation reduced code completion symbol search
         average latency from 16 ms to 1.09 ms for LLVM codebase (3M Lines Of
         Code) and 119 ms to 1.9 ms for Chromium codebase (16M LOC).
       \item Replaced previous Static Index implementation and gained over
         15x performance boost while providing more features for code
         completion-based symbol search.
       \item Implemented Variable length Byte (VByte) compression algorithm and
         reduced memory overhead by 60\%.
       \item Identified performance bottlenecks in commonly used LLVM YAML
         serializer and improved performance by a factor of 3.
     \end{itemize}}
  \cventry{2016}{Software Engineering
    Intern}{Google}{}{C++ Language Tools Team}{
    Worked on Clang-Rename, Clang-based tool for renaming symbols such as
    variables, functions, arguments.
    \begin{itemize}
      \item Fixed a number of existing issues, introduced new functionality
        and gained users for Clang-Rename.
      \item Built Vim and Emacs integration for Clang-Rename tool.
      \item Designed and prototyped
        \href{https://docs.google.com/document/d/1w9IkR0_Gqmd5w4CZ2t_ZDZrNLYVirQPyMS41533HQZE/edit?usp=sharing}
        {\textcolor{blue}{clang-refactor}}, which later resulted in
        \href{https://clang.llvm.org/docs/RefactoringEngine.html}{\textcolor{blue}{Clang's
        Refactoring Engine}}.
      \item Added new checks to Clang-Tidy and reduced false-positive rate for
        few existing checks.
    \end{itemize}}
  \cventry{2015}{Google Summer of Code student}{LLVM
    Community}{Remote}{}{Implemented Code Clone Detection tool in Clang Static
    Analyzer and used it to detect over 400 similar code pieces in Git,
    Vim, OpenSSL and other projects. GitHub page contains an
    \href{https://github.com/kirillbobyrev/code-clone-detection-llvm-devmtg15-poster\#code-clone-detection-llvm-devmtg15-poster}{\textcolor{blue}{overview}}
    of key results.}

\section{Open Source Contributions}
  \cventry{2015--Present}{LLVM}{Open Source Compiler Infrastructure Project}{}{}
    {Participating in LLVM-related open source projects and working on
     Clang-based C++ tooling. This includes contributions to Clangd, Clang-Tidy,
     LLVM Core Libraries and Clang Static Analyzer.}

\section{Skills}
  \cvdoubleitem{Languages}{Modern C++, Python}{Technologies}{LLVM, PyTorch,
    TensorFlow, OpenCV}

\nocite{*}
\bibliographystyle{plain}
\bibliography{publications}

\end{document}
